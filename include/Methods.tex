\chapter{Method}
We shall here describe our methodology for the development process of the protocol and the reference implementation, and provide backgrounds and motivations for the third party software used. The test network used for the reference implementation is also described.

\section{Implementation workflow}
All produced software artifacts and their source code have been open source throughout the development process, so that others might benefit from the research, especially since this technology is in its cradle. Dependencies on third party software should only be made following an assessment of the stability, security and longevity of the software. Based on these principles, the stability of the foundation upon which the produced artifacts depend can be maximized. Exempt from this principle was the Ethereum client AlethZero, which was under development throughout the entire length of the study. 

\section{General purpose blockchain}
An existing blockchain is used due to the inherent security in a wider network. Implementing a separate general purpose blockchain would be out of scope for the thesis. Thus, we have opted to utilize the Ethereum general purpose blockchain, and develop smart contracts for the Ethereum platform.

\section{Development tools}
The following tools have been used to provide a more efficient development process and a refined project planning structure.

% Borttaget då det inte är av relevans
%\subsection{Git}
%Git is a distributed version control system widely used by developers. It is based on speed, integrity of data and supported for distributed non-linear work flows~\cite{git}. Git is used in this project to manage code divided into different repositories and branches and to provide detailed logging of changes. 

\subsection{Pivotal Tracker}
In order to aid the agile development process Pivotal Tracker is used. Pivotal Tracker is a story-based planning tool which allows members of the team to add stories to a virtual project board. It is based on agile development methods, which suits the team. Pivotal Tracker provides several story logs: tasks for the current sprint, a backlog, and an icebox for stories that need grooming~\cite{pivotaltracker}. Stories are automatically arranged in the backlog according to their estimated size and the projected sprint velocity. Each story item can be further elaborated by a number of subtasks related to it. 

\subsection{AlethZero}
AlethZero is a proof of concept graphical client for Ethereum~\cite{github-alethzero}, implemented in C++ and Qt. It is used as a development environment for Ethereum-based decentralized applications and provides rich access to the blockchain and the DEVP2P network. For debugging, or manual blockchain interaction, it is possible to make Ether transactions, contract creation transactions or contract interaction transactions from the graphical user interface. A graphical user interface is also available for Whisper. AlethZero exposes the JSON-RPC API, making it possible to use it as a backend for decentralized applications.

\section{Test environment}
A test environment was chosen and set up in order to test the implementation of the cloud computing platform, with multiple actors. In this chapter detailed information about the system is provided.

\subsection{Hardware}
To test the project, a computer with sufficient processing capabilities to host a simple and small blockchain, as well as a web server when acting as a worker, was required. The Raspberry Pi 2 Model B was chosen because it is a cheap and viable computer. The hardware specifications are listed in Table~\ref{rpi}.

\begin{table}[h]
\centering
\caption{Raspberry Pi 2 Model B Specifications~\cite{rpi}.}
\label{rpi}
\begin{tabular}{|l|l|}
\hline \textbf{Processor} & 900MHz quad-core ARM Cortex-A7 CPU \\ \hline
\textbf{Memory} & 1,024 MB LPDDR2 \\ \hline 
\textbf{I/O Ports} & 4x USB, RJ45, microSD, 40-pin GPIO \\ \hline
\textbf{A/V Ports} & HDMI, RCA video, 3.5 mm audio \\ \hline
\textbf{Networking} & 10/100Mbit Ethernet \\ \hline
\textbf{Storage} & microSD card slot \\ \hline
\end{tabular}
\end{table}

\subsection{Network}
During development, for testing and researching purposes, a local network consisting of three Raspberry Pi 2 computers was deployed. This network has served as the backbone for distributed compilation, a private blockchain network, local Ethereum test network nodes, and client as well as worker emulation for the reference implementation called Zeppelin.

\textbf{Compiler network.} Since Ethereum is still unfinished software, it must be updated regularly. This takes a large amount of processing power every time it is done, and while the Raspberry Pi 2 is a capable machine for testing purposes, the several hour long task of recompiling Ethereum every other day is too much of an obstruction for the work flow. In order to speed up the process a distributed compiler network was deployed at an early stage. To facilitate this the hardware is networked together via an Ethernet switch, enabling fast communication between all the nodes as well as the Internet when required. 

\textbf{Private blockchain.} While testing and developing the reference implementation, some level of control over the blockchain was required. Primarily because both running as well as testing any software in the Ethereum network requires the use of the Ether cryptocurrency. For this reason, a local Ethereum blockchain was deployed during the early stages of development.

\textbf{Nodes in the public blockchain.} The test environment was eventually moved to the public blockchain, to have a complete mirror of the production environment while developing.

\textbf{Client and worker emulation.} Since access to decentralized applications on the Ethereum network can be provided through a web application, the clients running on the Raspberry Pi 2 computers can be accessed remotely and used as test slaves. The most simple way to achieve this, is to expose the JSON-RPC port, and setting the web application to use the remote JSON-RPC as its data source.
