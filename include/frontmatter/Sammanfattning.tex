
\thispagestyle{plain}			% Supress header 
\section*{Sammanfattning}
Denna rapport sätter decentraliserade nätverk i ett tekniskt och praktiskt sammanhang, samt beskriver en grund för decentraliserade molnbaserade beräkningsplattformar genom att föreslå ett protokoll som kan nyttjas för att skapa sådana plattformar. Huvudproblemet innefattar hur arbete ska distribueras och verifieras inom det decentraliserade nätverket. För att avvärja attacker riktade mot nätverket från både interna och externa aktörer används en kombination av incitament och ekonomiska principer. Protokollet använder både en generaliserad blockkedja och andra mer etablerade elektroniska kommunikationssätt. Genom att använda smarta kontrakt möjliggörs tillitslösa överenskommelser mellan deltagare i nätverket.

En referensimplementation av protokollet, vid namn \textit{Zeppelin}, har delvis realiserats i Ethereum-nätverkets generella blockkedja, och har testats på ett mindre nätverk. Referensimplementationen är modulär och demonstrerar möjligheten att använda både en traditionell backend och en blockkedjebaserad backend i decentraliserade applikationer. Genom att använda en blockkedja kan viss applikationsdata och affärslogik lagras och exekveras på en globalt distribuerad virtuell maskin. Referensimplementationen är ännu inte redo för driftsättning, utan ska ses som en teknikdemonstration av det föreslagna protokollet. Att referensimplementationen inte är redo för drift kan tillskrivas det faktum att generella blockkedjor i sitt nuvarande tillstånd ännu inte går att lita på i en produktionsmiljö, eftersom de fortfarande är under utveckling.

\newpage				% Create empty back of side
\thispagestyle{empty}
\mbox{}