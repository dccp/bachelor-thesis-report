
\thispagestyle{plain}			% Supress header 
\section*{Abstract}
This thesis describes a basis for decentralized cloud computing platforms; a suggested protocol to be used for such a platform; and places decentralized networks and their origins in a technical and practical context. The problem domain includes how to distribute work and how to verify computations within a decentralized network. To mitigate attacks on the network by internal and external actors, a combination of incentives and free market principles are used. The protocol is utilizing a general-purpose blockchain as well as other more established forms of network communication. Arbitration between network participants is realized using smart contracts.

A reference implementation of the protocol has been partially realized in the Ethereum general-purpose blockchain, and deployed on a small-scale network. The reference implementation is highly modular and demonstrates the ability for decentralized applications to use both a traditional backend and a blockchain-based backend. By using a blockchain, some application data and business logic is stored and executed on a global virtual machine, distributed between participating nodes. The reference implementation should be regarded as a proof-of-concept of the proposed protocol, and is not yet ready for a production release. This is largely attributed to the fact that general-purpose blockchains are currently in a very early development phase and can not yet be used reliably.


% KEYWORDS (MAXIMUM 10 WORDS)
\vfill
\noindent Keywords: decentralization, blockchain, trustless, consensus, Ethereum, smart, \\contract, Solidity. 

\newpage				% Create empty back of side
\thispagestyle{empty}
\mbox{}