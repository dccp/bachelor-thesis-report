\chapter{Conclusion}
In the introduction, three research questions were stated: How can work be distributed in a decentralized network? How can the network verify that workers execute work correctly? How can workers receive payment for the work they perform? To answer these questions, solve their associated problems and create a robust and reliable network for arbitrary cloud computation, both a general protocol and a modular reference implementation have been created. While the reference implementation only handles the problem of distributing work in the network, the proposed protocol should be seen as a generalized attempt at solving all three problems. 

In its current state, the reference implementation is unsuitable for use in a production environment due to security concerns. Even so, it is completely capable as a proof-of-concept for a decentralized application running in a general-purpose blockchain. In the future, reference implementation could be extended and further refined to implement the full protocol, which however is dependent on the advancement of the Ethereum decentralized application stack. A more rigid specification and verification of the protocol could be necessary to achieve a fully functional system.

The aim of the thesis is to eliminate the third party present in cloud computing services, thus reducing the need for trust. It is possible that the ongoing trend of such decentralization could be the next major advancement in the software sphere, and we are eager to both be a part of and watch it happen.

Throughout the study it has become obvious that the underlying general-purpose technologies currently available for decentralized applications are still in a very early development stage. Before a platform for decentralized cloud computing could be fully realized in such an environment, those dependencies must become more stable.
