\chapter{Results}
\textbf{Vad är intressant, och vad har vi tillfört? Sätt in i ett sammanhang.}

We have developed a decentralized system where an open market is used to distribute work throughout the network. The protocol is best suited to host various web applications, but could easily be extended to more complex types of applications. The protocol itself has no single point of failure, or external authority which might revoke access to it. However, the workers in the network can still be individually attacked, but it can be argued that workers have the motivation to mitigate such attacks as they only will be rewarded if they perform the work they set out to.
\\\\
It achieves decentralization because
\begin{inparaenum}[\itshape a\upshape)]
\item the specification, and reference implementation, are both open source and available from GitHub;
\item each end user retains a copy of the blockchain in which the contracts and data of the application are stored; and
\item each end user runs the application locally. 
\end{inparaenum}

The required aspects for the network to be transparent, resilient and trustless. By virtue of the open and distributed data store of blockchain, the network can be deemed to be transparent. Resilience is achieved on the data store level by the Ethereum network, and on the work execution level by incentives offered to workers. The inherent trustlessness in the blockchain and smart contracts allows parties to agree upon work and payment. We thus conclude that all requirements have been met.

All code for the reference implementation is available from https://github.com/dccp.